\chapter{Introduction}
\label{chap:introduction}

Personal computers have been developed to a point where those unfamiliar with Computer Science Theory might conclude there is nothing computers cannot do.
While this is an understandable conclusion, it has been proven that there is a limit to the types of computation our "classical computers" can perform \cite{linz}.
In the last few decades, however, the field of quantum mechanics and quantum computing have advanced to the point where primitive operations are now possible in the quantum sphere.
This breakthrough will change the way information is stored, secured, and created.
One such development was the BB84 quantum key distribution protocol.

There are many encryption protocols in use today, such as the Diffe-Hellman key exchange protocol (KEP), each of which provide their own various security measures.
The BB84 protocol allows two parties to co-generate an encryption key that can be used as a one-time-pad and discarded after use.
Throughout the process of key generation, the protocol gives a number of guarantees regarding eavesdropping and malicious acts to both parties \cite{qcftgu}.
This thesis presents a Python library that supports a peer to peer simulation of the BB84 quantum Quantum KEP, as well as serves as an intruduction to programming in the quantum computing paradigm.


Chapter~\ref{chap:background} of the thesis provides background information on encryption protocols, the quantum computation paradigm, and other background information related to the BB84 protocol. 

Chapter~\ref{chap:bb84} details the BB84 protocol in its steps, as well as discusses the guarantees the protocol provides the user.

Chapter~\ref{chap:implementation} describes the library.
This describes not only the documentation for the library but also how it can be used.

Chapter~\ref{chap:conclusion} summarizes this simulator and its potential applications, as well as discussing possible future work to improve upon it.

