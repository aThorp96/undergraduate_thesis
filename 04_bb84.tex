\chapter{Quantum Key Distribution}
\label{chap:bb84}

% Talk about quantum key distribution
Quantum computers have been shown to solve the underlying mathematical problems, such as the DLP and prime factorization that keep our encryption secure.
Although symmetric encryption is much less effected by this, it is not commonly used due to the key distribution problem \cite{cryptography}.
One potential solution is Quantum Key Distribution (QKD), which applies quantum mechanics and quantum computing properties to a key distribution protocol that is provably secure \cite{MikeAndIke}.
It allows for two parties to securely generate a symmetric key.
In this chapter, we look at the BB84 quantum key exchange protocol to exchange encryption keys between two parties, Alice and Bob.
The encryption key is exchanged through a quantum channel in the form of qubits.
An eavesdropper listening in on this communication cannot obtain any information without disturbing the qubits and measuring them, which introduces noise to the signal since the basis used to encode information is unknown. 
We show that such an eavesdropper can be detected by Alice and Bob due to a spike in the error rate of transmitted qubits.

% Introduce protocol
\section{The BB84 Protocol}

\begin{figure}[htp]
	\centering
	\fbox{\begin{minipage}{41em}
	\begin{center}
		\textbf{BB84}
	\end{center}
	\begin{enumerate}
	\item  Alice and Bob connect via a quantum communication channel.
    \item  Alice generates $N$ random bits, where $N \geq 4l$, $l$ = desired key length.
    \item  Alice encodes the bits into qubits, randomly choosing to encode using one of two orthonormal bases -- the standard and the Hadamard basis.
    \item  Alice sends the qubits to Bob.
    \item  Bob measures each qubit, choosing to use one of the two selected bases at random.
    \item  Bob sends a bitvector of his chosen bases to Alice over a classical channel.
    \item  Alice responds with a bitvector showing which of Bob's bases were correct.
    \item  Alice and Bob discard all qubits that were measured in the wrong basis.
    \item  Bob sends Alice some number of the measured values to ensure the key was received without any interference.
    \item Alice responds with whether or not the exchanged values were correct.
    \item If there were no errors in the compared bits, the bits that were exchanged are discarded by both parties, and the rest of the bits are used as the key.
    \end{enumerate}
	\end{minipage}}
\caption{AThe BB84 Quantum Key Distribution Protocol.}
	\label{fig:BB84}
\end{figure}

In 1984 Charles Bennett and Gilles Brassard proposed the first quantum key distribution protocol, the BB84 \cite{qc:agi}.
The protocol allows two parties communicating over a public classical channel, such as the internet, and a public quantum channel to securely generate a shared encryption key for symmetric encryption.

To start the quantum key exchange, Alice randomly generates two bit sequences, $a$ and $b$, of length at least $N = 4l$ bits each, where $l$ is the intended key length.
While $N$ does not necessarily need to be $4l$, it allows for $l$ bits to be used to verify the keys integrity, since half of the bits are discarded during the key exchange, on average \cite{MikeAndIke}.
Alice then encodes $a$ into a block of $N$ qubits, $\ket{\psi}$.
This is done using two bases; in our case, the standard basis and Hadamard basis.
The basis chosen for encoding each bit in $a$ is determined by the corresponding bit in $b$, with $b_i = \{0 \textrm{ or } 1\}$ where $0$ refers to the standard basis and $1$  refers to the Hadamard basis.
Thus, each qubit is in either the standard basis or the Hadamard basis, and is in one of four states shown in Figure ~\ref{fig:possible_states}.
\begin{figure}[htp]
\centering
\begin{tabular}{|c|c|c|}
\hline
$a_i$ & $b_i$ & $\ket{\psi_i}$ \\ \hline
0 & 0 & $\ket{0}$ \\ \hline
1 & 0 & $\ket{1}$ \\ \hline
0 & 1 & $\ket{+}$ \\ \hline
1 & 1 & $\ket{-}$ \\ \hline
\end{tabular}
\caption{The four possible states of a qubit in a BB84 encoded string.}
\label{fig:possible_states}
\end{figure}
Next, Alice sends each qubit to Bob using a public quantum channel.
When Bob has received each qubit he can assemble the full qubit block $\ket{\psi}^\prime$.
Assuming a perfect quantum channel and no eavesdropping, there should not be any disturbance or noise in the communication; therefore $\ket{\psi}^\prime = \ket{\psi}$.
Once Bob has received all qubits, he measures each qubit in $\ket{\psi}^\prime$ into a bit sequence $a^\prime$ by randomly choosing a basis of measurement for each bit.
The bases chosen are stored in a bit sequence $b^\prime$.
Bob then informs Alice that he has measured all the received qubits.
Because there is a 50\% chance that Bob will choose an incorrect basis for measurement for each qubit, and a 50\% probability of measuring the correct value using the wrong basis, Bob has measured 75\% of the qubits correctly, on average (see Figure ~\ref{fig:possible_measurements_no_eve}).
\begin{figure}[htp]
	\centering
	\begin{tabular}{|r|c|c|c|c|c|c|c|c|}
		\hline
		Encoded value  & 0 & 0 & 0 & 0 & 1 & 1 & 1 & 1 \\ \hline
		Encoded basis  & 0 & 0 & 1 & 1 & 0 & 0 & 1 & 1 \\ \hline
		Measured basis  & 0 & 1 & 0 & 1 & 0 & 1 & 0 & 1 \\ \hline
		Measured value & 0 & $\nicefrac{0}{1}$& $\nicefrac{0}{1}$ & 0 & 1 & $\nicefrac{0}{1}$ & $\nicefrac{0}{1}$ & 1\\
		\hline
	\end{tabular}
	\caption{Exhaustive list of encoding and measurement of a single qubit between Alice and Bob.}
	\label{fig:possible_measurements_no_eve}
\end{figure}
While 75\% of the bits are measured correct on average, an average of 50\% are measured in the correct basis; those qubits are  guaranteed to be measured correctly.
Note that Alice has ($a$, $b$) and Bob has($a^\prime$ $b^\prime$), but they each do not know the others $a$ and $a^\prime$.
Alice and Bob now exchange the bases they each used, $b$ and $b^\prime$, respectively.
Alice and Bob both discard every bit that was encoded and measured using different bases: bit $i$ is discarded from $a$ and $a^\prime$ if $b_i \neq b^\prime_i$.
They store the remaining bits from $a$ and $a^\prime$ into a new bit sequence, $k$ and $k^\prime$, respectively.
Alice and Bob now each have the same key, $k = k^\prime$.
They each exchange some number of bits from $k$ to verify there was no error in their key generation.
If the exchanged bits are identical, then Alice and Bob can confirm with high probability that there was no eavesdropper, and that their key is secure \cite{MikeAndIke}.
They can now use the key for symmetrically encrypted communication, or even as a OTP on a classical channel.

% Discuss eavesdropping
If an eavesdropper, Eve, were to attempt to listen in on the conversation, she would not be able to gain any useful information from the qubits, since she does not know the basis in which the qubits are encoded, nor could she duplicate the qubits. 
Therefore the only strategy she has is to perform a ``man-in-the-middle" attack, in which Eve impersonates Bob to Alice and Alice to Bob \cite{qc:agi}.

To eavesdrop on the communication, Eve listens on the quantum channel and waits for Alice to transmit qubits.
As Alice sends Bob the qubits over the quantum channel Eve intercepts each qubit, forming her own qubit block $\ket{\psi}^\prime$.
With the qubits now in Eve's possession, she attempt to measure the qubits or clone them on either of the two bases randomly.
She then re-encodes them using the same bases, into $\ket{\psi}^{\prime\prime}$, and forwards the qubits to Bob.
However, as previously shown, this would introduce noise to the signal, reducing Bob's correct average measurement.
Since the average percentage of correctness is $\frac{2 * 100\% + 6 * 50\%}{8} = 62.5\%$, the average percentage that a bit is measured correctly drops from 75\% to 62.5\%, leaving only 25\% of the qubits measured in the same basis used during encoding.
\begin{figure}[htp]
	\centering
	\begin{tabular}{|c|c|c|c|c|}
		% Table headers
		\hline
		$\textrm{Basis}_{Alice}$ & $\textrm{Basis}_{Eve}$ &$\textrm{Basis}_{Bob}$ & Percent Correct & Bit Kept in Key\\ \hline
	%	Ab  Eb  Bb  c       a       
    	0 & 0 & 0 & 100\% & Kept \\ \hline		
    	0 & 0 & 1 &  50\% & Discarded \\ \hline		
    	0 & 1 & 0 &  50\% & Kept \\ \hline		
    	0 & 1 & 1 &  50\% & Discarded \\ \hline		
    	1 & 0 & 0 &  50\% & Discarded \\ \hline		
    	1 & 0 & 1 &  50\% & Kept \\ \hline		
    	1 & 1 & 0 &  50\% & Discarded \\ \hline		
    	1 & 1 & 1 & 100\% & Kept \\ \hline		
	\end{tabular}
	\caption{Exhaustive list of encoding and measurement of a single qubit between Alice and Bob with an eavesdropper}
	\label{fig:possible_measurements_eve}
\end{figure}
As can be seen in figure~\ref{fig:possible_measurements_eve}, when there is an eavesdropper, half of the bits kept in the generated key are measured in an incorrect basis by at least one party.
Once the bits that Bob measured in a different basis than Alice are discarded, there is still only a $\frac{2 * 100\% + 2 * 50\%}{4} = 75\%$ chance of any qubit being measured by Bob as the same value encoded by Alice.
This means, on average, 25\% of the bits in $k^\prime$ are incorrect.
When Alice and Bob exchange some bits to verify their correctness, even if only four bits are compared they both will, on average, detect that the qubits were maliciously measured during transmission, at which point Alice and Bob can abort communication \cite{MikeAndIke}. 

In practice this protocol can be implemented using polarized photons as qubits, which can be sent between Alice and Bob using fiber optics.
The data is encoded into the photons using the angle of the polarization since photons can act as a qubit \cite{qc:agi}.
In this case the bases for encoding data are the standard basis, $(\ket{0}, \ket{1}) = (\ket{\rightarrow}, \ket{\uparrow})$, and the Hadamard basis, $(\ket{+}, \ket{-}) = (\ket{\nearrow}, \ket{\nwarrow})$.
However all that is required to perform any QKD protocol is the ability to communicate qubits over a public channel with a very low error rate \cite{MikeAndIke}.
