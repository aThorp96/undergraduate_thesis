\begin{abstract}

Although general purpose computers are quite powerful, there are limitations to the types of computations they can perform.
These limitations are exploited to provide a way of information security through encryption, such that it is very difficult, often infeasible, for a malicious party to obtain secured data.
Quantum computers however, have been shown to theoretically be able to perform these computations efficiently.
Certain encryption techniques are less susceptible to this issue, but they require both parties exchange a shared encryption key for secure communication.
In this thesis, we simulate the BB84 quantum key distribution protocol that allows two parties to securely exchange encryption keys.
Theoretically, this key exhcange protocol is highly reliable and secure, as it operates on the principles of quantum mechanics.

The BB84 quantum key distribution protocol uses a quantum communication channel to exchange qubits, and a classical channel that is used for verification purposes.
The security of the protocol comes from the fact that the basis used to measure qubits are exchanged after the transmission of the qubits.
Therefore, a malicious eavesdropper listening in to both channels will obtain no knowledge of the information exchanged. 

In this thesis, we simulate the BB84 quantum key distribution protocol using Python libraries Simulaqron and CQC.
We also develop a chat application that uses this protocol to exchange a symmetric key, communicates in a classical channel with information encrypted by the encryption key, and is capable of detecting if a malicious adversary was listening during the key exchange process.
Additionally, we give a brief introduction to quantum computation and the Python simulation libraries in hopes that this thesis serves as reference material for students interested in getting started programming in the paradigm.


\end{abstract}
